Nowadays, cyber-physical systems (CPS), that integrate computations and communications with physical processes, are receiving high popularity and being widely adopted in various application areas such as healthcare, water treatment, power grid, transportation, military, robotics, and so forth.
% However, the main design challanges in CPS are: one side, the increasing prevalence of cyber attacks targeting them; the other side, its nature of delay sensitivity. 
However, the increasing prevalence of cyber attacks targeting them poses a growing security concern. Besides deception attacks, that mostly target communication channel vulnerabilities, firmware-based attacks are another widely common cyber attacks in CPS. This class of attacks mainly exploit memory safety vulnerabilities, e.g., buffer overflows and dangling pointers. These vulnerabilities are common in CPS firmwares, e.g., PLC firmwares, because of the fact that those firmwares are implemented with C/C++ languages for performance and many other reasons.\\ %Besides being  vulnerabilities for cyber attacks, memory safety violations could also cause severe damage on the system by resulting system crashes and silent data corruptions.\\ 

%The main vulnerabilities being exploited by this class of attacks are memory safety violations, e.g., buffer overflows and dangling pointers.

%One hand, those memory safety tools did not fit for the complex cps environment as they are not dedicatedly implemented for that purpose, on the otherhand, these tools incur high runtime overhead which might not be tolerable in CPS. 

As a countermeasure, wide range of memory safety tools have been developed that dynamically detect and mitigate memory safety violations. However, these tools incur high runtime overheads, which might not be tolerable in most delay-sensitive and real-time constrained cyber-physical systems. Delay could result severe damage in the system or open a security hole for delay-based attacks. This essentially makes performance as critical as security in most CPS. \\ 

On this research, we empirically measure memory safety overheads for CPS and quantify tolerability of CPS on such overheads. By doing so, we can able to 1) specify the minimum hardware requirements, e.g., processor speed, that a given cyber-physical system needs to have to work normally with a specific memory safety tool or 2) identify the memory safety tool whose runtime overhead does not affect real-time constraints of the cyber-physical system. 
