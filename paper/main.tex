%\documentclass[12pt,journal,compsoc]{IEEEtran}
%\documentclass[twocolumn]{article}
%{IEEEtran}
%\usepackage[total={6.0in,8.0in},top=0.8in,
 % left=1.3in, includefoot]{geometry}
\documentclass{acm_proc_article-sp}
%\documentclass{sig-alternate-05-2015}
%\usepackage{epsfig,setspace,subfig,url,amsmath, booktabs}
%\usepackage{graphicx}
%\usepackage{caption}
\usepackage{float}
\usepackage{hyperref} 

%new packages
\usepackage{verbatim}

%For author
%\newcommand*{\affaddr}[1]{#1} % No op here. Customize it for different styles.
\newcommand*{\affmark}[1][*]{\textsuperscript{#1}}
%\newcommand*{\email}[1]{\texttt{#1}}


\newcommand{\nextitem}{\par\hspace*{\labelsep}\textbullet\hspace*{\labelsep}}
%\renewcommand{\familydefault}{\sfdefault}

\begin{document}
%\pdfpageheight 12in 
%\pdfpagewidth 14in
\title{Empirically Measuring Memory Safety Overheads for CPS}
\subtitle{[Draft]}
\begin{comment}
\numberofauthors{4}
\author{
% 1st. author
\alignauthor
Mart\'{i}n Ochoa\\
       \affaddr{Singapore University of Technology and Design}\\
       %\affaddr{University of Padua}\\
       \email{martin\_ochoa@sutd.edu.sg}
% 2nd. author
\alignauthor
Eyasu Getahun Chekole\\
       \affaddr{Singapore University of Technology and Design}\\
       \affaddr{Advanced Digital Sciences Center, Illinois at Singapore}\\
       \email{eyasu\_chekole@mymail.sutd.edu.sg}\\
\and
% 3rd. author
\alignauthor
David K. Y. Yau\\
       \affaddr{Singapore University of Technology and Design}\\
       \affaddr{Advanced Digital Sciences Center, Illinois at Singapore}\\
       \email{david\_yau@sutd.edu.sg}
% 4th. author
\alignauthor
John Henry Castellanos Alvarado\\
       \affaddr{Singapore University of Technology and Design}\\
       %\affaddr{Via Garibaldi 63 - 25087, Salo (BS), IT}\\
       \email{john\_alvarado@mymail.sutd.edu.sg}
}
\end{comment}
\begin{comment}
\author{%
Mart\'{i}n Ochoa\affmark[1], Eyasu Getahun Chekole\affmark[1,2], John Henry Alvarado\affmark[1], David K. Y. Yau\affmark[1,2]\\\\
\affaddr{\affmark[1]Singapore University of Technology and Design}\\
\email{\{martin\_ochoa, david\_yau\}@sutd.edu.sg}\\
\email{\{eyasu\_chekole, john\_alvarado\}@mymail.sutd.edu.sg}\\\\
\affaddr{\affmark[2]Advanced Digital Sciences Center, Illinois at Singapore}\\
\email{\{eyasu.c, david.yau\}@adsc.com.sg}
%\affaddr{\LaTeX\ University}%
}
\end{comment}

\begin{comment}
\author[1]{Alice Smith}
\author[2]{Bob Jones}
\affil[1]{Department of Mathematics, University X}
\affil[2]{Department of Biology, University Y}
\end{comment}

%\date{\vspace{-5ex}}
\maketitle

\begin{abstract}
The abstract goes here ...
Nowadays, cyber-physical systems (CPS), that integrate computations and communications with physical processes, are receiving high popularity and being widely adopted in various application areas such as healthcare, water treatment, power grid, transportation, military, robotics, and so forth.
% However, the main design challanges in CPS are: one side, the increasing prevalence of cyber attacks targeting them; the other side, its nature of delay sensitivity. 
However, the increasing prevalence of cyber attacks targeting them poses a growing security concern. Besides deception attacks, that mostly target communication channel vulnerabilities, firmware-based attacks are another widely common cyber attacks in CPS. This class of attacks mainly exploit memory safety vulnerabilities, e.g., buffer overflows and dangling pointers. These vulnerabilities are common in CPS firmwares, e.g., PLC firmwares, because of the fact that those firmwares are implemented with C/C++ languages for performance and many other reasons.\\ %Besides being  vulnerabilities for cyber attacks, memory safety violations could also cause severe damage on the system by resulting system crashes and silent data corruptions.\\ 

%The main vulnerabilities being exploited by this class of attacks are memory safety violations, e.g., buffer overflows and dangling pointers.

%One hand, those memory safety tools did not fit for the complex cps environment as they are not dedicatedly implemented for that purpose, on the otherhand, these tools incur high runtime overhead which might not be tolerable in CPS. 

As a countermeasure, wide range of memory safety tools have been developed that dynamically detect and mitigate memory safety violations. However, these tools incur high runtime overheads, which might not be tolerable in most delay-sensitive and real-time constrained cyber-physical systems. Delay could result severe damage in the system or open a security hole for delay-based attacks. This essentially makes performance as critical as security in most CPS. \\ 

On this research, we empirically measure memory safety overheads for CPS and quantify tolerability of CPS on such overheads. By doing so, we can able to 1) specify the minimum hardware requirements, e.g., processor speed, that a given cyber-physical system needs to have to work normally with a specific memory safety tool or 2) identify the memory safety tool whose runtime overhead does not affect real-time constraints of the cyber-physical system. 

\end{abstract}

%\noindent \textbf{\Large{Categories and Subject Descriptors}}\\
%\noindent D.4.6 \textbf{[Security and Privacy Protection]}: Access controls\\\\
%\noindent \textbf{\Large{General Terms}}\\
%\noindent Security\\\\
%\noindent \textbf{\Large{Keywords}} \\
%\noindent Data privacy, Data security, Access control models, PoGAC

%\category{D.4.6}{Security and Privacy Protection}{Access controls}
%\terms{Security}
%\keywords{Data privacy, Data security, Access control models, PoGAC}

\section{Introduction}\label{sec:introduction}
Introduction of the paper will include the following points:
\begin{enumerate}
\item \textbf{Problem description}
\begin{itemize}
\item Clear description of memory safety vulnerabilities in CPS (specifically in context of SWaT) - \emph{preliminary problem - P0}.
\item Explaining (class of) attacks that target those vulnerabilities (firmware-based attacks, in general) - \emph{initial problem - P1}.
\item High level explanation of memory safety solutions: specifically the MS tool we are using (its features, reason of choosing it, etc). 
\item Elaborating memory safety overheads and how it affects a CPS plant, e.g., in SWaT - \emph{main problem and focus of the paper - P2}.
\end{itemize}
\item \textbf{Our proposed solution} 
\begin{itemize}
\item Introduce a memory safety tool in CPS environment, which is
\begin{itemize}
\item Effective in security (in detecting memory errors) - hence solving \emph{P1}.
\item Efficient in performance (tolerable in the system) - hence solving \emph{P2}.
\end{itemize}
\end{itemize}
\item \textbf{Contributions}: contributions (novelty claims) of our work. 
\begin{itemize}
\item Adopting memory safety solutions in the CPS environment - \emph{referring the evidence on Section xx}.
\item Benchmarking and empirically measuring its runtime overhead (MSO) in CPS environment - \emph{referring the evidence on Section xx}.
\item Quantifying and ensuring its tolerability (before it creates any disruption to the system) - \emph{referring the evidence on Section xx}.
\end{itemize}

\item \textbf{Structure of the paper}: explaining how our paper is organized. 
\begin{itemize}
\item Section 2: xx
\item Section 3: xx
\item Section 4: xx
\item Etc.
\end{itemize}
\end{enumerate}

\section{Background}\label{background}
\hspace{1cm} [Optional section] \footnote{Could be merged with Section \ref{systemmodel}}
\begin{itemize}
\item Characteristics of a CPS environment
\begin{itemize}
\item Ever lasting process (could result buffer overflows for unbounded buffers)
\item Severe resource constraints, e.g., in PLCs: limited memory and CPU speed
\item Hard real-time constraints
\end{itemize}
\item Overview of SWaT (details will be included in Section \ref{systemmodel})
\begin{itemize}
\item Overview
\item Architecture
\item etc
\end{itemize}
\item Address sanitizer
\begin{itemize}
\item How it works - a compile-time instrumentation tool.
\item Memory errors coverage
\item Performance
\item Reason to choose it
\end{itemize}
\item OpenPLC
\item ScadaBR
\end{itemize}

\section{System model: SWaT}\label{systemmodel}
\begin{enumerate}
\item \textbf{General architecture of the plant}
\begin{itemize}
\item Internal architecture of the system 
\begin{itemize}
\item Hardware architecture of the PLC, e.g., processor, registers, memory, etc.
\item The PLC scan cycle 
\end{itemize}
\item Interacting components
\begin{itemize}
\item Sensors, actuators
\item Other PLCs
\item SCADA systems
\item Communication topology
\item Interaction constraints, e.g., frequency, accessing shared resources (e.g., mutual exclusion and the implication of locking on performance), etc.
\end{itemize}
\end{itemize}
\item \textbf{Security concerns of the plant}: 
\begin{itemize}
\item Focusing on memory safety vulnerabilities.
\item Highlighting attack vectors (and exploits) targeting those vulnerabilities. 
\end{itemize}
\item \textbf{Performance constraints}
\begin{itemize}
\item Real-time constraints (hard-deadline) of the system
\item Elaborating how the runtime overhead, i.e, MSO, can affect the plant (e.g., violating the real-time constraint). 
\end{itemize}
\end{enumerate}

\section{Empirically measuring memory safety overheads}\label{sec:measuremso}
\begin{enumerate}
\item Benchmark design
\item Measurement details, e.g., implementation.
%\item Main factors/variables 
\item Measure MSO of each individual PLC operations: scan inputs, program execution, update Outputs, others (locking/unlocking threads). 
\item Measure overall MSO
\item Show if there is exceptional deviation in performance 
\begin{itemize}
\item MSO is an average value, but a single exceptional scan time could disrupt the system. 
\item Hence, needs to graphically show variance of each scan time
\end{itemize}
\end{enumerate}

\section{Quantify tolerability}\label{sec:tolerability}
\begin{itemize}
\item Identifying and characterizing performance constraints of the system
\begin{itemize}
\item Internal constraints, e.g., cycle time.
\item External constraints, e.g., frequency of communications with external components.
\end{itemize}
\item Quantifying a tolerable MSO range.
\end{itemize}

\section{Evaluation}
\begin{itemize}
\item Security evaluation
\begin{itemize}
\item Coverage
\item Effectiveness
\end{itemize}
\item Performance evaluation (referring Section \ref{sec:measuremso} and \ref{sec:tolerability})
\begin{itemize}
\item Analyzing MSO and its tolerability 
\end{itemize}
\end{itemize}
\section{Related work}
\begin{itemize}
\item Security and memory safety related works in CPS or embedded systems. 
\item Performance overhead related works in CPS environment. 
\end{itemize}
\section{Conclusion and future works}
\begin{itemize}
\item Conclusion
\item Future works
\end{itemize}
\bibliographystyle{abbrv}
\bibliography{mybib}

%Show complete list of references.
\nocite{*}
\end{document}
